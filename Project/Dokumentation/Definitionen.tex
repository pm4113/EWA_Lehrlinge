%\ifthenelse{\boolean{english}}
%{}{
%	\usepackage{german}
%}



%wird ben�tigt bei Mehrle skripten
\usepackage{etex}


\usepackage[tmargin=1in,bmargin=1in,lmargin=1.25in,rmargin=1.25in]{geometry}
\usepackage{titlesec}
\usepackage{xcolor}
\usepackage[overload]{textcase}
\definecolor{MSBlue}{rgb}{.204,.353,.541}
\definecolor{MSLightBlue}{rgb}{.31,.506,.741}
% Set formats for each heading level
%\titleformat*{\section}{\rmfamily\bfseries\huge\color{MSBlue}\lowercase}
\titleformat{\section}[hang]{\rmfamily\bfseries\huge\color{MSBlue}}{\thesection}{1em}{}[]
\titleformat{\subsection}{\large\bfseries\sffamily}{\thesubsection}{1em}{}
\titleformat{\subsubsection}{\sffamily\bfseries}{\thesubsubsection}{1em}{}


\titleformat{\chapter}
{\rmfamily\bfseries\Large\huge} % format\Large\bfseries
{}                % label
{0pt}             % sep
{\huge}           % before-code




%\renewcommand\contentsname{\ifthenelse{\boolean{english}}{table of contents.}{inhaltsverzeichnis.}}
%\renewcommand\listfigurename{\ifthenelse{\boolean{english}}{list of figures.}{abbildungsverzeichnis.}}
%\renewcommand\listtablename{\ifthenelse{\boolean{english}}{list of tables.}{tabellenverzeichnis.}}

%\renewcommand\refname{\litname.}
\usepackage{exscale,relsize}
\usepackage{fancyhdr}
\usepackage[small]{caption}
\usepackage{units}
\usepackage{subfigure}
\usepackage{wallpaper}
\usepackage{rotating}
\usepackage[]{amsmath}
\usepackage{mathrsfs}
\usepackage{makeidx}
%\usepackage{pgfplots}
\usepackage{tikz}

\usepackage{import}

%\usepackage{amssymb}
%\usepackage{pst-all}
%\usepackage{amsmath}
%\usepackage{float}


\usepackage{listings}
\usepackage{color} 
\lstset{ 
language=bash, 
basicstyle=\ttfamily,
commentstyle=\ttfamily, 
numbersep=5pt, 
backgroundcolor=\color{white}, 
showspaces=false, 
showstringspaces=false,
breaklines=true,
postbreak=\raisebox{0ex}[0ex][0ex]{\ensuremath{\color{black}\hookrightarrow\space}},
frame=single, 
tabsize=2
} 



%\usepackage[pdftex,bookmarks=true,bookmarksnumbered=true]{hyperref}

%\newcommand{\babel}[2]{\ifthenelse{\boolean{english}}{#1}{#2}}

% color set
%\definecolorseries{foo}{rgb}{last}[rgb]{1.0,0.0,0.0}[rgb]{0.0,0.0,1.0}
%\resetcolorseries[16]{foo}

% auxillary symbols
\renewcommand{\tilde}{\symbol{126}}
\newcommand{\define}{\stackrel{!}{=}}
\renewcommand{\equiv}{\,\widehat{=}\,}
\newcommand{\subsubsubsection}{\textbf}
\newcommand{\re}{\mathrm{Re}}
\newcommand{\pr}{\mathrm{Pr}}
\newcommand{\st}{\mathrm{St}}
\newcommand{\fr}{\mathrm{Fr}}
\newcommand{\nus}{\mathrm{Nu}}
\newcommand{\gr}{\mathrm{Gr}}
\newcommand{\ra}{\mathrm{Ra}}
\newcommand{\mif}{\quad\mathrm{\babel{if}{falls}}\quad}
\newcommand{\with}{\quad\mathrm{\babel{with}{mit}}\quad}
\newcommand{\for}{\quad\mathrm{\babel{for}{f"ur}}\quad}
%\renewcommand{\not}{\not}
\newcommand{\im}{i}
\newcommand{\ariwam}{ARiWaM}
\newcommand{\matlab}{MATLAB}

% format specifications
\renewcommand{\emph}{\textbf}
\newcommand{\file}{\textit}
\newcommand{\cmd}{\texttt}
\newcommand{\ten}{\boldsymbol}

%\newcommand{\unit}{\mathrm}
\newcommand{\lemma}{\textit}
\newcommand{\deutsch}[1]{german: \textit{#1}}
\renewcommand{\index}{\emph}

% Command path to graphic files
\newcommand{\gpath}{./grafics}
\newcommand{\bsppath}{../uebungen/beispiele}

% mathematical operators
\newcommand{\grad}{\,\mathrm{grad}\,}
\renewcommand{\div}{\,\mathrm{div}\,}
\newcommand{\rot}{\,\mathrm{rot}\,}
\newcommand{\lap}{\Delta}
\newcommand{\laplace}[1]{\mathscr{L}\left\{#1\right\}}
\newcommand{\trans}{^T}
\newcommand{\norm}{\psarc[linewidth=0.5pt](0,0){0.4}{0}{90}\psdot[dotsize=0.1](0.15,0.15)}

% electrical networks
\newcommand{\con}{\pscircle[linewidth=2pt,fillstyle=solid,fillcolor=black](0,0){0.1}}
\newcommand{\pin}{\pscircle[linewidth=2pt,fillstyle=solid](0,0){0.15}}
\newcommand{\link}{\psline[linewidth=2pt]}
\newcommand{\linkd}{\psline[linestyle=dashed]}
\newcommand{\arcd}{\psarc[linestyle=dashed]}
\newcommand{\arrow}{\psline[linewidth=2pt,arrowsize=8pt]{->}}
\newcommand{\cs}{\pscircle[linewidth=2pt](0,0){0.5}\arrow(0,-0.4)(0,0.4)\link(0,-0.5)(0,-2)\link(0,0.5)(0,2)}
\newcommand{\vs}{\pscircle[linewidth=2pt](0,0){0.5}\psline(-0.2,-0.1)(0.2,-0.1)\psline(-0.2,0)(0.2,0)\pscurve(-0.2,0.1)(-0.1,0.15)(0,0.1)(0.1,0.05)(0.2,0.1)\link(0,-0.5)(0,-2)\link(0,0.5)(0,2)}
\newcommand{\res}{\psframe[linewidth=2pt](-0.3,-1)(0.3,1)\link(0,-1)(0,-2)\link(0,1)(0,2)}
\newcommand{\ind}{\psframe[linewidth=2pt,fillstyle=solid,fillcolor=black](-0.3,-1)(0.3,1)\link(0,-1)(0,-2)\link(0,1)(0,2)}
\renewcommand{\cap}{\link(-0.5,-0.1)(0.5,-0.1)\link(-0.5,0.1)(0.5,0.1)\link(0,-0.1)(0,-2)\link(0,0.1)(0,2)}
\newcommand{\opv}{\pspolygon[linewidth=2pt](-1,-1.3)(-1,1.3)(1.3,0)\link(-1,-0.8)(-2,-0.8)\link(-1,0.8)(-2,0.8)\link(1.3,0)(2,0)\rput(-0.7,0.8){$-$}\rput(-0.7,-0.8){$+$}}

% mechanical networks
\newcommand{\mass}{\psframe[linewidth=2pt](-0.5,-0.5)(0.5,0.5)\link(0,0.5)(0,1)\link(0,-0.5)(0,-1)}
\newcommand{\spring}{\link(0,-1)(0,-0.8)(-0.3,-0.7)(0.3,-0.5)(-0.3,-0.3)(0.3,-0.1)(-0.3,0.1)(0.3,0.3)(-0.3,0.5)(0.3,0.7)(0,0.8)(0,1)}
\newcommand{\damp}{\link(-0.5,0.4)(-0.5,-0.3)(0.5,-0.3)(0.5,0.4)\link(-0.42,0.2)(0.42,0.2)\link(0,0.2)(0,1)\link(0,-0.3)(0,-1)}
\newcommand{\hatch}{\psframe[fillstyle=vlines,linestyle=none,hatchwidth=0.25pt]}
\newcommand{\clamp}{\hatch(-0.5,-0.3)(0.5,0)\link(-0.5,0)(0.5,0)}
\newcommand{\fixed}{\hatch(-0.5,-0.7)(0.5,-0.4)\link(-0.5,-0.4)(0.5,-0.4)\link(-0.25,-0.4)(0,0)(0.25,-0.4)\pscircle[linewidth=2pt,fillstyle=solid](0,0){0.1}}
\newcommand{\loose}{\hatch(-0.5,-0.7)(0.5,-0.5)\link(-0.5,-0.4)(0.5,-0.4)\link(-0.5,-0.5)(0.5,-0.5)\link(-0.25,-0.4)(0,0)(0.25,-0.4)\pscircle[linewidth=2pt,fillstyle=solid](0,0){0.1}}
\newcommand{\axial}{\link(-0.3,-0.3)(-0.3,-0.1)(0.3,-0.1)(0.3,-0.3)\link(-0.3,0.3)(-0.3,0.1)(0.3,0.1)(0.3,0.3)}
\newcommand{\force}{\psline[linewidth=2pt,arrowsize=6pt]{->}}
\newcommand{\rmom}{\psarc[arrowsize=6pt,linewidth=2pt]{<-}}
\newcommand{\lmom}{\psarc[arrowsize=6pt,linewidth=2pt]{->}}
\newcommand{\stress}{\psline[arrowsize=6pt]{->}}
\newcommand{\beam}{\psframe[fillstyle=solid,fillcolor=lightgray]}
\newcommand{\rod}{\psline[linewidth=3pt]}
\newcommand{\disp}{\psline[linewidth=2pt,arrowsize=6pt]{->}}	% displacement

% hydraulical networks
\newcommand{\pump}{\pscircle[linewidth=2pt,fillstyle=solid,fillcolor=white](0,0){1}\pspolygon[fillstyle=solid,fillcolor=black](0,1)(-0.15,0.3)(0.15,0.3)}
\newcommand{\motor}{\pscircle[linewidth=2pt,fillstyle=solid,fillcolor=white](0,0){0.8}\rput(0,0){M}}
\newcommand{\throttle}{\psarc[linewidth=2pt](-3,0){2.7}{-15}{15}\psarc[linewidth=2pt](3,0){2.7}{165}{195}}
\newcommand{\limiter}{\psframe[linewidth=2pt,fillstyle=solid,fillcolor=white](-0.7,-0.7)(0.7,0.7)\psline{->}(-0.5,-0.4)(-0.5,0.1)(0.5,-0.25)}

% diagrams (block and s-plane)
\newcommand{\axis}{\psline[arrowsize=6pt,arrowinset=0]{->}}
\newcommand{\measure}{\psline[linewidth=0.5pt,arrowsize=4pt]}
\newcommand{\xtick}[1]{\measure(#1,-0.1)(#1,0.1)}
\newcommand{\ytick}[1]{\measure(-0.1,#1)(0.1,#1)}
\newcommand{\ang}{\psarc[linewidth=0.5pt,arrowsize=4pt]{<->}}
\newcommand{\tf}[1]{\psframe[linewidth=2pt](-1,-0.75)(1,0.75)\rput(0,0){#1}}
\newcommand{\tff}[2]{\psframe[fillstyle=solid,fillcolor=lightgray](-1,0)(1,0.75)\psframe[linewidth=2pt](-1,-0.75)(1,0.75)\rput(0,0.35){#1}\rput(0,-0.35){#2}}
\newcommand{\sat}{\psframe[linewidth=2pt](-1,-0.75)(1,0.75)\link(-0.8,-0.5)(-0.5,-0.5)(0.5,0.5)(0.8,0.5)}
\newcommand{\add}[4]{\pscircle[linewidth=2pt,fillstyle=solid,fillcolor=white](0,0){0.5}\rput(-0.5,-0.75){#1}\rput(-0.75,0.5){#2}\rput(0.5,0.75){#3}\rput(0.75,-0.5){#4}}
\newcommand{\pole}{\link(-0.15,-0.15)(0.15,0.15)\link(-0.15,0.15)(0.15,-0.15)}
\newcommand{\zero}{\pscircle[linewidth=2pt,fillstyle=solid](0,0){0.15}}
\newcommand{\textblock}[1]{\psframebox[linewidth=2pt]{\parbox[c][2cm][c]{3cm}{\centering #1}}}

% thermodynamic systems
\newcommand{\cv}{\psframe[linestyle=dashed]}
\newcommand{\sys}{\psframe[fillcolor=white,fillstyle=solid,linewidth=2pt]}
\newcommand{\iso}{\psframe[linewidth=2pt,fillstyle=vlines]}
\newcommand{\candle}{\psframe[linewidth=0.5pt](-0.15,-0.5)(0.15,0.3)\pscurve[linewidth=0.5pt](0,0.8)(-0.1,0.5)(0,0.3)(0.1,0.5)(0,0.8)\pscurve[linewidth=0.5pt](0,0.6)(-0.08,0.4)(0,0.3)(0.08,0.4)(0,0.6)}
\newcommand{\flux}{\psline[arrowsize=10pt,linewidth=4pt]{->}}
\newcommand{\point}[1]{\pscircle(0,0){0.3}\rput[c](0,0){#1}}

% aerodynamics
\newcommand{\source}{\pin\rput{0}(0,0){\arrow(0.5,0)(1,0)}\rput{45}(0,0){\arrow(0.5,0)(1,0)}\rput{90}(0,0){\arrow(0.5,0)(1,0)}\rput{135}(0,0){\arrow(0.5,0)(1,0)}\rput{180}(0,0){\arrow(0.5,0)(1,0)}\rput{225}(0,0){\arrow(0.5,0)(1,0)}\rput{270}(0,0){\arrow(0.5,0)(1,0)}\rput{315}(0,0){\arrow(0.5,0)(1,0)}}

% CFD
\newcommand{\cell}[1]{\definecolor{cellc}{gray}{#1}\psframe[linewidth=2pt,fillstyle=solid,fillcolor=cellc](0,0)(1,1)}

%\renewcommand{\labelenumi}{\alph{enumi})}

\setlength{\parindent}{0em}
\setlength{\parskip}{1.5ex plus0.5ex minus0.5ex}
\setlength{\captionmargin}{3em}

% counters
%\newcounter{example}
%\ifthenelse{\boolean{english}}{\newcommand{\exampletext}{example }}{\newcommand{\exampletext}{Beispiel }}
%\newcommand{\example}[1]{\underline{\exampletext \arabic{chapter}.\arabic{example}:} #1 \addtocounter{example}{1}}
%\newcounter{exercise}
%\ifthenelse{\boolean{english}}{\newcommand{\exercisetext}{exercise }}{\newcommand{\exercisetext}{Aufgabe }}
%\newcommand{\exercise}[1]{\underline{\exercisetext \arabic{chapter}.\arabic{exercise}:} #1 \stepcounter{exercise}}
%\newcommand{\cchapter}[1]{\chapter{#1} \setcounter{example}{1} \setcounter{exercise}{1}}
%\ifthenelse{\boolean{english}}{\newcommand{\solution}{\textit{solution:} }}{\newcommand{\solution}{\textit{L\"osung:} }}
%%\newcounter{beispiel}
%%\setcounter{beispiel}{1}		% Nummer des ersten Beispiels
%\newboolean{student}
%\newcounter{enumcount}
%%\newcommand{\resume}[1]{\begin{#1} \setcounter{enumi}{\value{enumcount}}}
%%\newcommand{\pause}[1]{\setcounter{enumcount}{\value{enumi}} \end{#1}}


\usepackage{hyperref}


% deutsche Anpassungen
%\usepackage[ansinew]{inputenc}
\usepackage[T1]{fontenc}
\usepackage[utf8]{inputenc}
\usepackage[ngerman]{babel}
%\usepackage[ngerman, english]{babel}

%\usepackage{babelbib}

% mathematische Symbole
\usepackage{amsmath,amssymb,amsfonts,amstext}

% erweiterte Zeichenbefehle
\usepackage{pst-all}

% Kopfzeilen frei gestaltbar
\usepackage{fancyhdr}
\lfoot[\fancyplain{}{}]{\fancyplain{}{}}
\rfoot[\fancyplain{}{}]{\fancyplain{}{}}
\cfoot[\fancyplain{}{\footnotesize\thepage}]{\fancyplain{}{\footnotesize\thepage}}
\lhead[\fancyplain{}{\footnotesize\nouppercase\leftmark}]{\fancyplain{}{}}
\chead{}
\rhead[\fancyplain{}{}]{\fancyplain{}{\footnotesize\nouppercase\sc\leftmark}} 

% Farben im Dokument m"oglich
\usepackage{color}

% Schriftart Helvetica
\usepackage{helvet}
\renewcommand{\familydefault}{cmss}

% anderdhalbfacher Zeilenabstand
\usepackage{setspace}
\onehalfspacing



% Graphiken einbinden 
\usepackage{graphicx} 

\usepackage{epstopdf} 
%\usepackage{epsfig}
\usepackage{psfrag}




%\usepackage{pstool}


%\usepackage{pgfplots}

%\usepackage{pdfpages}
%\usepackage{pstricks, pst-node, pst-plot, pst-circ}
%\usepackage{moredefs}

% verbesserte Floating Plazierung
\usepackage{float}

% "Uberpr"ufung des Layouts
\usepackage{layout}

\usepackage{array}

% erweiterte Einstellungen der Bildunterschriften -> 8 Pt
%\usepackage[small]{caption}

\usepackage{ifthen}

% H"ohe und Breite des Textk"orpers etwas gr"osser definieren
\usepackage[tmargin=1in,bmargin=1in,lmargin=1.25in,rmargin=1.25in]{geometry}

% Einr"uckung von und Abstand zwischen Abs"atzen
\setlength{\parindent}{0em}
\setlength{\parskip}{1.5ex plus0.5ex minus0.5ex}

% weniger Warnungen wegen "uberf"ullter Boxen
\tolerance = 9999
\sloppy

% Anpassung einiger "Uberschriften 
%\renewcommand\figurename{Abbildung}
%\renewcommand\tablename{Tabelle}

% Counter f"ur die Nummerierung
\newcounter{romancount}

% Boolsche Variable f"ur Bachelor-/Masterarbeit oder Bericht
\newboolean{thesis}

%Aufzählungen spalten
\usepackage{multicol} 
